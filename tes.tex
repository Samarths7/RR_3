% Options for packages loaded elsewhere
\PassOptionsToPackage{unicode}{hyperref}
\PassOptionsToPackage{hyphens}{url}
%
\documentclass[
]{article}
\usepackage{amsmath,amssymb}
\usepackage{lmodern}
\usepackage{iftex}
\ifPDFTeX
  \usepackage[T1]{fontenc}
  \usepackage[utf8]{inputenc}
  \usepackage{textcomp} % provide euro and other symbols
\else % if luatex or xetex
  \usepackage{unicode-math}
  \defaultfontfeatures{Scale=MatchLowercase}
  \defaultfontfeatures[\rmfamily]{Ligatures=TeX,Scale=1}
\fi
% Use upquote if available, for straight quotes in verbatim environments
\IfFileExists{upquote.sty}{\usepackage{upquote}}{}
\IfFileExists{microtype.sty}{% use microtype if available
  \usepackage[]{microtype}
  \UseMicrotypeSet[protrusion]{basicmath} % disable protrusion for tt fonts
}{}
\makeatletter
\@ifundefined{KOMAClassName}{% if non-KOMA class
  \IfFileExists{parskip.sty}{%
    \usepackage{parskip}
  }{% else
    \setlength{\parindent}{0pt}
    \setlength{\parskip}{6pt plus 2pt minus 1pt}}
}{% if KOMA class
  \KOMAoptions{parskip=half}}
\makeatother
\usepackage{xcolor}
\usepackage[margin=1in]{geometry}
\usepackage{color}
\usepackage{fancyvrb}
\newcommand{\VerbBar}{|}
\newcommand{\VERB}{\Verb[commandchars=\\\{\}]}
\DefineVerbatimEnvironment{Highlighting}{Verbatim}{commandchars=\\\{\}}
% Add ',fontsize=\small' for more characters per line
\usepackage{framed}
\definecolor{shadecolor}{RGB}{248,248,248}
\newenvironment{Shaded}{\begin{snugshade}}{\end{snugshade}}
\newcommand{\AlertTok}[1]{\textcolor[rgb]{0.94,0.16,0.16}{#1}}
\newcommand{\AnnotationTok}[1]{\textcolor[rgb]{0.56,0.35,0.01}{\textbf{\textit{#1}}}}
\newcommand{\AttributeTok}[1]{\textcolor[rgb]{0.77,0.63,0.00}{#1}}
\newcommand{\BaseNTok}[1]{\textcolor[rgb]{0.00,0.00,0.81}{#1}}
\newcommand{\BuiltInTok}[1]{#1}
\newcommand{\CharTok}[1]{\textcolor[rgb]{0.31,0.60,0.02}{#1}}
\newcommand{\CommentTok}[1]{\textcolor[rgb]{0.56,0.35,0.01}{\textit{#1}}}
\newcommand{\CommentVarTok}[1]{\textcolor[rgb]{0.56,0.35,0.01}{\textbf{\textit{#1}}}}
\newcommand{\ConstantTok}[1]{\textcolor[rgb]{0.00,0.00,0.00}{#1}}
\newcommand{\ControlFlowTok}[1]{\textcolor[rgb]{0.13,0.29,0.53}{\textbf{#1}}}
\newcommand{\DataTypeTok}[1]{\textcolor[rgb]{0.13,0.29,0.53}{#1}}
\newcommand{\DecValTok}[1]{\textcolor[rgb]{0.00,0.00,0.81}{#1}}
\newcommand{\DocumentationTok}[1]{\textcolor[rgb]{0.56,0.35,0.01}{\textbf{\textit{#1}}}}
\newcommand{\ErrorTok}[1]{\textcolor[rgb]{0.64,0.00,0.00}{\textbf{#1}}}
\newcommand{\ExtensionTok}[1]{#1}
\newcommand{\FloatTok}[1]{\textcolor[rgb]{0.00,0.00,0.81}{#1}}
\newcommand{\FunctionTok}[1]{\textcolor[rgb]{0.00,0.00,0.00}{#1}}
\newcommand{\ImportTok}[1]{#1}
\newcommand{\InformationTok}[1]{\textcolor[rgb]{0.56,0.35,0.01}{\textbf{\textit{#1}}}}
\newcommand{\KeywordTok}[1]{\textcolor[rgb]{0.13,0.29,0.53}{\textbf{#1}}}
\newcommand{\NormalTok}[1]{#1}
\newcommand{\OperatorTok}[1]{\textcolor[rgb]{0.81,0.36,0.00}{\textbf{#1}}}
\newcommand{\OtherTok}[1]{\textcolor[rgb]{0.56,0.35,0.01}{#1}}
\newcommand{\PreprocessorTok}[1]{\textcolor[rgb]{0.56,0.35,0.01}{\textit{#1}}}
\newcommand{\RegionMarkerTok}[1]{#1}
\newcommand{\SpecialCharTok}[1]{\textcolor[rgb]{0.00,0.00,0.00}{#1}}
\newcommand{\SpecialStringTok}[1]{\textcolor[rgb]{0.31,0.60,0.02}{#1}}
\newcommand{\StringTok}[1]{\textcolor[rgb]{0.31,0.60,0.02}{#1}}
\newcommand{\VariableTok}[1]{\textcolor[rgb]{0.00,0.00,0.00}{#1}}
\newcommand{\VerbatimStringTok}[1]{\textcolor[rgb]{0.31,0.60,0.02}{#1}}
\newcommand{\WarningTok}[1]{\textcolor[rgb]{0.56,0.35,0.01}{\textbf{\textit{#1}}}}
\usepackage{graphicx}
\makeatletter
\def\maxwidth{\ifdim\Gin@nat@width>\linewidth\linewidth\else\Gin@nat@width\fi}
\def\maxheight{\ifdim\Gin@nat@height>\textheight\textheight\else\Gin@nat@height\fi}
\makeatother
% Scale images if necessary, so that they will not overflow the page
% margins by default, and it is still possible to overwrite the defaults
% using explicit options in \includegraphics[width, height, ...]{}
\setkeys{Gin}{width=\maxwidth,height=\maxheight,keepaspectratio}
% Set default figure placement to htbp
\makeatletter
\def\fps@figure{htbp}
\makeatother
\setlength{\emergencystretch}{3em} % prevent overfull lines
\providecommand{\tightlist}{%
  \setlength{\itemsep}{0pt}\setlength{\parskip}{0pt}}
\setcounter{secnumdepth}{-\maxdimen} % remove section numbering
\ifLuaTeX
  \usepackage{selnolig}  % disable illegal ligatures
\fi
\IfFileExists{bookmark.sty}{\usepackage{bookmark}}{\usepackage{hyperref}}
\IfFileExists{xurl.sty}{\usepackage{xurl}}{} % add URL line breaks if available
\urlstyle{same} % disable monospaced font for URLs
\hypersetup{
  hidelinks,
  pdfcreator={LaTeX via pandoc}}

\author{}
\date{\vspace{-2.5em}}

\begin{document}

\hypertarget{synopsis}{%
\subsection{1: Synopsis}\label{synopsis}}

The goal of the assignment is to explore the NOAA Storm Database and
explore the effects of severe weather events on both population and
economy.The database covers the time period between 1950 and November
2011.

The following analysis investigates which types of severe weather events
are most harmful on:

\begin{enumerate}
\def\labelenumi{\arabic{enumi}.}
\tightlist
\item
  Health (injuries and fatalities)
\item
  Property and crops (economic consequences)
\end{enumerate}

\hypertarget{data-processing}{%
\subsection{2: Data Processing}\label{data-processing}}

\hypertarget{data-loading}{%
\subsubsection{2.1: Data Loading}\label{data-loading}}

Download the raw data file and extract the data into a dataframe.Then
convert to a data.table

\begin{Shaded}
\begin{Highlighting}[]
\NormalTok{library("ggplot2")}

\NormalTok{fileUrl \textless{}{-} "https://d396qusza40orc.cloudfront.net/repdata\%2Fdata\%2FStormData.csv.bz2"}
\NormalTok{download.file(fileUrl, destfile = paste0("/Users/mgalarny/Desktop", \textquotesingle{}/repdata\%2Fdata\%2FStormData.csv.bz2\textquotesingle{}))}
\NormalTok{stormDF \textless{}{-} read.csv("/Users/mgalarny/Desktop/repdata\%2Fdata\%2FStormData.csv.bz2")}

\NormalTok{\# Converting data.frame to data.table}
\NormalTok{stormDT \textless{}{-} as.data.table(stormDF)}
\end{Highlighting}
\end{Shaded}

\hypertarget{examining-column-names}{%
\subsubsection{2.2: Examining Column
Names}\label{examining-column-names}}

\begin{verbatim}
colnames(stormDT)
\end{verbatim}

\hypertarget{data-subsetting}{%
\subsubsection{2.3: Data Subsetting}\label{data-subsetting}}

Subset the dataset on the parameters of interest. Basically, we remove
the columns we don't need for clarity.

\begin{verbatim}
# Finding columns to remove
cols2Remove <- colnames(stormDT[, !c("EVTYPE"
  , "FATALITIES"
  , "INJURIES"
  , "PROPDMG"
  , "PROPDMGEXP"
  , "CROPDMG"
  , "CROPDMGEXP")])

stormDT[, c(cols2Remove) := NULL]

stormDT <- stormDT[(EVTYPE != "?" & 
             (INJURIES > 0 | FATALITIES > 0 | PROPDMG > 0 | CROPDMG > 0)), c("EVTYPE"
                                                                            , "FATALITIES"
                                                                            , "INJURIES"
                                                                            , "PROPDMG"
                                                                            , "PROPDMGEXP"
                                                                            , "CROPDMG"
                                                                            , "CROPDMGEXP")
\end{verbatim}

\hypertarget{converting-exponent-columns-into-actual-exponents-instead-of---h-k-etc}{%
\subsubsection{2.4: Converting Exponent Columns into Actual Exponents
instead of (-,+, H, K,
etc)}\label{converting-exponent-columns-into-actual-exponents-instead-of---h-k-etc}}

Making the PROPDMGEXP and CROPDMGEXP columns cleaner so they can be used
to calculate property and crop cost.

\begin{verbatim}
# Change all damage exponents to uppercase.
cols <- c("PROPDMGEXP", "CROPDMGEXP")
stormDT[,  (cols) := c(lapply(.SD, toupper)), .SDcols = cols]

# Map property damage alphanumeric exponents to numeric values.
propDmgKey <-  c("\"\"" = 10^0,
                 "-" = 10^0, 
                 "+" = 10^0,
                 "0" = 10^0,
                 "1" = 10^1,
                 "2" = 10^2,
                 "3" = 10^3,
                 "4" = 10^4,
                 "5" = 10^5,
                 "6" = 10^6,
                 "7" = 10^7,
                 "8" = 10^8,
                 "9" = 10^9,
                 "H" = 10^2,
                 "K" = 10^3,
                 "M" = 10^6,
                 "B" = 10^9)

# Map crop damage alphanumeric exponents to numeric values
cropDmgKey <-  c("\"\"" = 10^0,
                "?" = 10^0, 
                "0" = 10^0,
                "K" = 10^3,
                "M" = 10^6,
                "B" = 10^9)

stormDT[, PROPDMGEXP := propDmgKey[as.character(stormDT[,PROPDMGEXP])]]
stormDT[is.na(PROPDMGEXP), PROPDMGEXP := 10^0 ]

stormDT[, CROPDMGEXP := cropDmgKey[as.character(stormDT[,CROPDMGEXP])] ]
stormDT[is.na(CROPDMGEXP), CROPDMGEXP := 10^0 ]
\end{verbatim}

\hypertarget{making-economic-cost-columns}{%
\subsubsection{2.5: Making Economic Cost
Columns}\label{making-economic-cost-columns}}

\begin{verbatim}
stormDT <- stormDT[, .(EVTYPE, FATALITIES, INJURIES, PROPDMG, PROPDMGEXP, propCost = PROPDMG * PROPDMGEXP, CROPDMG, CROPDMGEXP, cropCost = CROPDMG * CROPDMGEXP)]
\end{verbatim}

\hypertarget{calcuating-total-property-and-crop-cost}{%
\subsubsection{2.6: Calcuating Total Property and Crop
Cost}\label{calcuating-total-property-and-crop-cost}}

\begin{verbatim}
totalCostDT <- stormDT[, .(propCost = sum(propCost), cropCost = sum(cropCost), Total_Cost = sum(propCost) + sum(cropCost)), by = .(EVTYPE)]

totalCostDT <- totalCostDT[order(-Total_Cost), ]

totalCostDT <- totalCostDT[1:10, ]

head(totalCostDT, 5)
\end{verbatim}

\hypertarget{calcuating-total-fatalities-and-injuries}{%
\subsubsection{2.7: Calcuating Total Fatalities and
Injuries}\label{calcuating-total-fatalities-and-injuries}}

\begin{verbatim}
totalInjuriesDT <- stormDT[, .(FATALITIES = sum(FATALITIES), INJURIES = sum(INJURIES), totals = sum(FATALITIES) + sum(INJURIES)), by = .(EVTYPE)]

totalInjuriesDT <- totalInjuriesDT[order(-FATALITIES), ]

totalInjuriesDT <- totalInjuriesDT[1:10, ]

head(totalInjuriesDT, 5)
\end{verbatim}

\hypertarget{results}{%
\subsection{3: Results}\label{results}}

\hypertarget{events-that-are-most-harmful-to-population-health}{%
\subsubsection{3.1: Events that are Most Harmful to Population
Health}\label{events-that-are-most-harmful-to-population-health}}

Melting data.table so that it is easier to put in bar graph format

\begin{verbatim}
bad_stuff <- melt(totalInjuriesDT, id.vars="EVTYPE", variable.name = "bad_thing")
head(bad_stuff, 5)
\end{verbatim}

\begin{verbatim}
# Create chart
healthChart <- ggplot(bad_stuff, aes(x=reorder(EVTYPE, -value), y=value))

# Plot data as bar chart
healthChart = healthChart + geom_bar(stat="identity", aes(fill=bad_thing), position="dodge")

# Format y-axis scale and set y-axis label
healthChart = healthChart + ylab("Frequency Count") 

# Set x-axis label
healthChart = healthChart + xlab("Event Type") 

# Rotate x-axis tick labels 
healthChart = healthChart + theme(axis.text.x = element_text(angle=45, hjust=1))

# Set chart title and center it
healthChart = healthChart + ggtitle("Top 10 US Killers") + theme(plot.title = element_text(hjust = 0.5))

healthChart
\end{verbatim}

\#\#\#3.2: Events that have the Greatest Economic Consequences Melting
data.table so that it is easier to put in bar graph format

\begin{verbatim}
econ_consequences <- melt(totalCostDT, id.vars="EVTYPE", variable.name = "Damage_Type")
head(econ_consequences, 5)
\end{verbatim}

\begin{verbatim}
# Create chart
econChart <- ggplot(econ_consequences, aes(x=reorder(EVTYPE, -value), y=value))

# Plot data as bar chart
econChart = econChart + geom_bar(stat="identity", aes(fill=Damage_Type), position="dodge")

# Format y-axis scale and set y-axis label
econChart = econChart + ylab("Cost (dollars)") 

# Set x-axis label
econChart = econChart + xlab("Event Type") 

# Rotate x-axis tick labels 
econChart = econChart + theme(axis.text.x = element_text(angle=45, hjust=1))

# Set chart title and center it
econChart = econChart + ggtitle("Top 10 US Storm Events causing Economic Consequences") + theme(plot.title = element_text(hjust = 0.5))

econChart
\end{verbatim}

\end{document}
